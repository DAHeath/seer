\documentclass{article}

\begin{document}

Vlad and I discussed two directions for potential research:
\textbf{(1)} branch elimination and \textbf{(2)} information leakage
understanding.

Both tasks seem highly linked to information flow.
%
For the first problem, consider a program which is partially evaluated
on some input. The task is to decide if it is possible to
delete a branch of the program without revealing addtional information
about the output that is not already implied by the output, or by some
leakage factor.
%
In the second problem, the task is to construct an oracle that can
answer yes/no questions about program leakage. The user provides an
input program and a property and the oracle indicates whether or not
the program leaks the property in question.

It seems like it is possible to encode both problems as CHC problems.
The second problem \emph{seems} more straightforward.
%
Potentially, it reduces directly to information flow: Given two
executions of the program, is it possible to observe the property in
question. TODO, how exactly does this work?

The second question seems harder but also possible. Given two
executions of the program -- one where the branch is taken and one
where it is not -- is the branch condition revealed?

\end{document}
