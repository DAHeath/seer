\documentclass{article}

\usepackage{fullpage}
\usepackage{amsmath}
\usepackage{tikz}

\begin{document}
Blind Seer is a database technology that supports multi-party
computation (MPC).
%
The database is a flat structure where each database record is
accessed by traversing a tree.
%
Each vertex in the tree contains the union of all labels of records
accessible from that vertex.
%
In order to access particular records according to an input query, the
client and the server follow an MPC protocol. By doing so, the server
prevents information about the database from propogating to the client
(and presumably the client prevents information about their query from
propogating to the server).

The algorithm for addressing a query traverses the search tree one
vertex at a time, deciding if each vertex \emph{satisfies} the query
(meaning there is information in the label of the vertex that matches
the query).
%
Let's assume that the two parties attempt to determine if some
arbitrary vertex satisfies the query.
%
Then, the client provides the query and the server provides the label for
the vertex. Together, they determine satisfaction via MPC.
%
If the vertex satisfies the query, then the two parties recurse on the
children of the given vertex.
%
Otherwise, the two parties terminate searching the current branch of
the tree.
%
This continues until the tree has been fully traversed, and any
records which were reached are propogated to the
client.

While the individual vertex computations are secured via MPC, the
overall traversal is not.
%
Therefore, some information is leaked.
%
By carefully constructing queries, it is possible to learn new
information about the content of the database which may be
undesirable.
%

One interesting problem space attempts to identify how information
leaks as a result of the structure of the tree traversals.
%
As a starting point, we know that a purely disjunctive query does not
leak any information beyond the query result. We know this because
there are no `partial branches' in the tree traversal.
%
Specifically, if any arbitrary vertex satisfies the query, then there
must be at least one path from that vertex to a matching record.
%
However, conjunctive queries do not have the same behavior. It is
possible that a given vertex succeeds in satisfying the query, yet
both of its children fail.
%
When this happens, some information has been leaked: The attacker
knows that there is a vertex that satisfies their query that does not
lead to a specific matching record.
%
However, qualifying what information has leaked is a difficult task.

We will employ Programming Language techniques to address
problems in this space.
%
The user \emph{may} be asked to provide some of the following
elements:

\begin{itemize}
\item The fully populated database.
\item A policy which describes what types of queries are allowed.
\item A query which asks if it is possible for certain information to
leak.
\end{itemize}

The oracle we will construct will use this information to construct
either \textbf{(1)} a proof that described information leakage is
impossible or \textbf{(2)} a query(ies) that leads to the information
leakage we wanted to prevent.
%
One possible direction for addressing such problems starts by
considering only single queries in isolation: Specifically, how
qualifying what information is leaked by a conjunctive query.
%
Then, this work could be extended to consider multiple queries.
\end{document}
